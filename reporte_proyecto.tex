\documentclass[12pt,a4paper]{article}
\usepackage[utf8]{inputenc}
% \usepackage[spanish]{babel} % Desactivado por compatibilidad
\usepackage{hyperref}
\usepackage{listings}
\usepackage{xcolor}
\usepackage{graphicx}
\usepackage{geometry}
\usepackage{fancyhdr}
\usepackage{titlesec}

% Configuración de márgenes y diseño
\geometry{margin=2.5cm}
\pagestyle{fancy}
\fancyhf{}
\rhead{\textbf{Laboratorio de Contraseñas}}
\lhead{Reporte Técnico}
\rfoot{Página \thepage}

% Colores para código
\definecolor{codegreen}{rgb}{0,0.6,0}
\definecolor{codegray}{rgb}{0.5,0.5,0.5}
\definecolor{codepurple}{rgb}{0.58,0,0.82}
\definecolor{backcolour}{rgb}{0.95,0.95,0.92}

\lstdefinestyle{mystyle}{
    backgroundcolor=\color{backcolour},   
    commentstyle=\color{codegreen},
    keywordstyle=\color{magenta},
    numberstyle=\tiny\color{codegray},
    stringstyle=\color{codepurple},
    basicstyle=\ttfamily\footnotesize,
    breakatwhitespace=false,         
    breaklines=true,                 
    captionpos=b,                    
    keepspaces=true,                 
    numbers=left,                    
    numbersep=5pt,                  
    showspaces=false,                
    showstringspaces=false,
    showtabs=false,                  
    tabsize=2
}

\lstset{style=mystyle}

\begin{document}

% Portada Personalizada
\begin{titlepage}
    \centering
    \vspace*{1cm}
    
    {\Huge \textbf{Laboratorio de Contraseñas}}
    
    \vspace{0.5cm}
    {\Large \textit{"El Hacker Amable"}}
    
    \vspace{1.5cm}
    
    \textbf{Reporte Técnico Completo}
    
    \vspace{1.5cm}
    
    \textbf{Equipo de Desarrollo:}
    \begin{itemize}
        \item Persona 1: Base de Datos
        \item Persona 2: Evaluador de Seguridad
        \item Persona 3: Almacenamiento
        \item Persona 4: Integración e Interfaz
    \end{itemize}
    
    \vfill
    
    \vspace{0.8cm}
    
    {\large \today}
    
\end{titlepage}

\tableofcontents
\newpage

\section{Resumen Ejecutivo}
El \textbf{Laboratorio de Contraseñas} es una herramienta educativa diseñada para analizar, evaluar y mejorar la seguridad de las contraseñas de los usuarios. A través de un enfoque amigable y un lenguaje coloquial ("chido"), buscamos concientizar sobre la importancia de la ciberseguridad sin abrumar con tecnicismos. El sistema no solo detecta contraseñas débiles, sino que ofrece métricas avanzadas como la entropía y herramientas proactivas como un generador de claves seguras.

\section{Arquitectura del Sistema}
El proyecto está construido en Python y sigue una arquitectura modular dividida en cuatro componentes principales, correspondientes a los roles del equipo.

\subsection{1. Base de Datos y Hashing (Persona 1)}
\textbf{Archivo:} \texttt{database.py}

Este módulo actúa como la primera línea de defensa. Su función principal es identificar contraseñas que ya han sido comprometidas o que son extremadamente comunes.

\begin{itemize}
    \item \textbf{Lista Negra Ampliada:} Contiene una extensa base de datos de contraseñas inseguras, incluyendo patrones numéricos ("123456"), secuencias de teclado ("qwerty", "asdf") y términos comunes en español ("hola", "teamo", "futbol").
    \item \textbf{Seguridad por Hashing:} Para proteger la integridad del análisis, las contraseñas no se almacenan en texto plano. Se utiliza el algoritmo \textbf{SHA-256} para generar una huella digital única (hash) de cada contraseña.
\end{itemize}

\subsection{2. Sistema de Evaluación (Persona 2)}
\textbf{Archivo:} \texttt{evaluator.py}

Es el motor analítico del sistema. Evalúa la fortaleza de una contraseña basándose en reglas heurísticas y matemáticas.

\begin{itemize}
    \item \textbf{Reglas de Puntuación:}
    \begin{itemize}
        \item Longitud $\ge$ 8 caracteres: +20 puntos.
        \item Uso de números: +15 puntos.
        \item Uso de símbolos especiales: +25 puntos.
        \item Uso de mayúsculas/minúsculas: +25 puntos.
    \end{itemize}
    \item \textbf{Penalización Total:} Si el hash de la contraseña coincide con alguno en la base de datos de \texttt{database.py}, la puntuación se reduce automáticamente a 0.
\end{itemize}

\subsection{3. Almacenamiento y Reportes (Persona 3)}
\textbf{Archivo:} \texttt{storage.py}

Gestiona la persistencia temporal y la generación de informes.

\begin{itemize}
    \item \textbf{Bitácora en Memoria:} Mantiene un registro de todas las contraseñas analizadas durante la sesión, clasificándolas en "Débiles" y "Seguras".
    \item \textbf{Estadísticas en Tiempo Real:} Calcula promedios de puntuación y conteos de vulnerabilidades encontradas.
    \item \textbf{Exportación de Datos:} Permite generar un archivo físico (\texttt{reporte\_seguridad.txt}) con el resumen de la sesión para su posterior revisión.
\end{itemize}

\subsection{4. Interfaz y Generador (Persona 4)}
\textbf{Archivos:} \texttt{password\_lab.py} y \texttt{generator.py}

Es el punto de entrada para el usuario y el orquestador de todos los módulos.

\begin{itemize}
    \item \textbf{CLI Interactiva:} Un menú de consola intuitivo que guía al usuario a través de las diferentes funcionalidades.
    \item \textbf{Integración Modular:} Coordina las llamadas a la base de datos, el evaluador y el almacenamiento.
    \item \textbf{Generador de Contraseñas:} Un módulo adicional que crea contraseñas robustas utilizando una mezcla criptográficamente segura de caracteres.
\end{itemize}

\section{Características Avanzadas}
Para llevar el proyecto al siguiente nivel, se implementaron funcionalidades técnicas avanzadas:

\subsection{Cálculo de Entropía}
Más allá de reglas simples, implementamos el cálculo de la \textbf{Entropía de Shannon}. Esta métrica mide la incertidumbre o "sorpresa" de la contraseña en bits.
\[ H = L \times \log_2(N) \]
Donde $L$ es la longitud de la contraseña y $N$ es el tamaño del conjunto de caracteres posibles. Una entropía alta significa que la contraseña es matemáticamente difícil de adivinar por fuerza bruta.

\subsection{Generación Automática}
El sistema incluye un generador que asegura la inclusión de todos los tipos de caracteres necesarios (mayúsculas, minúsculas, números, símbolos) para garantizar una alta entropía desde el origen.

\section{Análisis Visual de Datos}
Para comprender mejor el comportamiento de las contraseñas analizadas, hemos generado gráficos representativos basados en datos simulados del laboratorio.

\subsection{Distribución de Seguridad}
El siguiente gráfico muestra la proporción de contraseñas clasificadas por su nivel de seguridad. Como se observa, una gran parte de las contraseñas probadas suelen ser débiles o comunes.

\begin{figure}[h]
    \centering
    \includegraphics[width=0.7\textwidth]{grafico_distribucion.png}
    \caption{Distribución de tipos de contraseñas analizadas.}
    \label{fig:distribucion}
\end{figure}

\subsection{Comparativa de Entropía}
La entropía es una métrica clave. En este gráfico comparamos cómo aumenta la entropía (bits) a medida que la contraseña se vuelve más compleja (longitud, variedad de caracteres).

\begin{figure}[h]
    \centering
    \includegraphics[width=0.8\textwidth]{grafico_entropia.png}
    \caption{Entropía estimada según la complejidad de la contraseña.}
    \label{fig:entropia}
\end{figure}

\section{Guía de Usuario}

\subsection{Inicio}
Para iniciar el laboratorio, abra una terminal en la carpeta del proyecto y ejecute:
\begin{lstlisting}[language=bash]
python3 password_lab.py
\end{lstlisting}

\subsection{Menú Principal}
Verá las siguientes opciones:
\begin{enumerate}
    \item \textbf{Probar contraseña:} Ingrese una clave para ver su puntaje, entropía y si es común.
    \item \textbf{Ver historial:} Muestra las contraseñas analizadas en la sesión actual.
    \item \textbf{Ver estadísticas:} Resumen numérico del rendimiento de sus contraseñas.
    \item \textbf{Generar contraseña:} Crea una clave segura automáticamente.
    \item \textbf{Exportar reporte:} Guarda los resultados en un archivo de texto.
    \item \textbf{Salir:} Cierra el programa.
\end{enumerate}



\end{document}
